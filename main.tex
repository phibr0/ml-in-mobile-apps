\documentclass[12pt]{article}
\usepackage{graphicx}%Package um Grafiken einzufügen
\graphicspath{ {assets/} }
\usepackage[ngerman]{babel}%Sprache Deutsch einstellen
\usepackage[headheight=15pt, a4paper, left=40mm, top=20mm, bottom=20mm, right=20mm]{geometry}
%\usepackage[]{fontspec}
%\setmainfont{Inter.ttf}
\usepackage[]{fancyhdr}
\pagestyle{fancy}
%\setlength{\headheight}{16pt}
\lhead{\leftmark}
\rhead{Phillip Bronzel} 

\usepackage[style=authoryear, backend=biber]{biblatex}
\addbibresource{references.bib}
\usepackage{csquotes}

\title{Machine Learning in Smartphone Apps}
\date{Phillip Bronzel \today}
\author{ASGSG Informatik, 2020/2021}

\begin{document}
  \maketitle
  \pagenumbering{gobble}
  \begin{center}
   \includegraphics[totalheight=10cm]{titlepage.png}
   \cite{titlepageimage}
  \end{center}

  \newpage
  \vspace*{\fill}
  \newpage
\vspace*{\fill}
\begin{center}Hiermit versichere ich, dass ich die Arbeit selbstständig verfasst, dass ich keine anderen Quellen und Hilfsmittel als die angegebenen benutzt und die Stellen der Arbeit, die anderen Quellen dem Wortlaut oder Sinn nach entnommen sind, in jedem einzelnen Fall unter Angabe von Quellen kenntlich gemacht habe.
\end{center}
\vspace*{\fill}

  \vspace*{\fill}

  \newpage
  \pagenumbering{arabic}
  \tableofcontents

  \newpage
  \section{Einführung}
  Lorem Ipsum blah bla\footnote{\cite[vergleiche][Seite 666]{nnfs}}%TODO: Kapitel der App einfügen

  \newpage
  \printbibliography[heading=bibintoc, title={Literaturverzeichnis}]
\end{document}