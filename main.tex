\documentclass[12pt]{article}
\usepackage{graphicx}%Package um Grafiken einzufügen
\graphicspath{ {assets/} }
\usepackage[ngerman]{babel}%Sprache Deutsch einstellen
\usepackage[headheight=15pt, a4paper, left=4cm, top=2cm, bottom=2cm, right=2cm]{geometry}
\usepackage[]{fontspec}
\setmonofont{CascadiaCode.ttf}[Scale=0.88]
\usepackage[]{fancyhdr}
\pagestyle{fancy}
%\setlength{\headheight}{16pt}
\lhead{\leftmark}
\rhead{Phillip a Bronzel} 

\usepackage[style=authoryear, backend=biber]{biblatex}
\addbibresource{references.bib}
\usepackage{csquotes}

\usepackage{eso-pic}
\newcommand\BackgroundPic{
    \put(0,0){
    \parbox[b][\paperheight]{\paperwidth}{
    \vfill
    \centering
    \includegraphics[width=\paperwidth,height=\paperheight]{assets/titlebackground.pdf}
    \vfill
    }}}

\usepackage{chronology}

\usepackage{subfig}

\usepackage{pgfplots}

\usepackage[onehalfspacing]{setspace}

\usepackage[bottom]{footmisc}

\usepackage{wrapfig}

\usepackage{xcolor}

\usepackage{svg}

\usepackage{float}

\usepackage{minted}
\usemintedstyle{pastie}
\renewcommand\listoflistingscaption{Quellcodeverzeichnis}

\usepackage{amsmath}

\usepackage{tikz}
\usetikzlibrary{matrix,calc,arrows.meta,bending}

%%%%%%%%%%%%%%%%%%%%%%%%%%%%%%%%%%%%%%%%%%%%%%%%%%%%%%%%%%%%%%%%%%%%%%
% LaTeX Overlay Generator - Annotated Figures v0.0.1
% Created with http://ff.cx/latex-overlay-generator/
% If this generator saves you time, consider donating 5,- EUR! :-)
%%%%%%%%%%%%%%%%%%%%%%%%%%%%%%%%%%%%%%%%%%%%%%%%%%%%%%%%%%%%%%%%%%%%%%
%\annotatedFigureBoxCustom{bottom-left}{top-right}{label}{label-position}{box-color}{label-color}{border-color}{text-color}
\newcommand*\annotatedFigureBoxCustom[8]{\draw[#5,thick,rounded corners] (#1) rectangle (#2);\node at (#4) [fill=#6,thick,shape=circle,draw=#7,inner sep=2pt,text=#8] {\textbf{#3}};}
%\annotatedFigureBox{bottom-left}{top-right}{label}{label-position}
\newcommand*\annotatedFigureBox[4]{\annotatedFigureBoxCustom{#1}{#2}{#3}{#4}{white}{white}{black}{black}}
\newcommand*\annotatedFigureText[4]{\node[draw=none, anchor=south west, text=#2, inner sep=0, text width=#3\linewidth] at (#1){#4};}
\newenvironment {annotatedFigure}[1]{\centering\begin{tikzpicture}
\node[anchor=south west,inner sep=0] (image) at (0,0) { #1};\begin{scope}[x={(image.south east)},y={(image.north west)}]}{\end{scope}\end{tikzpicture}}
%%%%%%%%%%%%%%%%%%%%%%%%%%%%%%%%%%%%%%%%%%%%%%%%%%%%%%%%%%%%%%%%%%%%%%

\usepackage{ifthen}

\setcounter{secnumdepth}{3} \setcounter{tocdepth}{3}

\title{Machine Learning in Smartphone Apps}
\date{Phillip Bronzel \today}
\author{ASGSG Informatik, 2020/2021}

\begin{document}
\AddToShipoutPicture*{\BackgroundPic}
\maketitle
\pagenumbering{gobble}
\begin{center}
    \includegraphics[totalheight=10cm]{titlepage.png}
    \cite{titlepageimage}
\end{center}

\newpage
\vspace*{\fill}
\begin{center}Hiermit versichere ich, dass ich die Arbeit selbstständig verfasst, dass ich keine anderen Quellen und Hilfsmittel als die angegebenen benutzt und die Stellen der Arbeit, die anderen Quellen dem Wortlaut oder Sinn nach entnommen sind, in jedem einzelnen Fall unter Angabe von Quellen kenntlich gemacht habe.
\end{center}
\vspace*{\fill}


\newpage
\pagenumbering{arabic}
\tableofcontents

\newpage
Lorem Ipsum blah bla\footnote{\cite[vergleiche][Seite 666]{nnfs}}%TODO: Kapitel der App einfügen

\section{Neuronale Netzwerke}

\subsection{Geschichte}

Im Jahr 1943 wurde die erste Arbeit darüber geschrieben, wie Neuronen im Gehirn funktionieren könnten und die Autoren Warren McCulloch und Walter Pitts experimentierten sogar damit diese mit elektronischen Schaltkreisen nachzubauen.\footnote{\cite[]{alogicalcalculus}}

In den 1950er Jahren haben Forscher von IBM daran gearbeitet ein Neuronales Netzwerk mit einem Computer zu simulieren. Der Versuch scheiterte allerdings.\footnote{\cite[Absatz 3]{nnhistory}}

Immer wieder gab es kleinere Forschungsprojekte, ein sehr großer Durchbruch war aber 1975 die Entwicklung eines "`Backpropagation"' Algorithmus durch den Wissenschaftler Paul Werbos. Ähnliche Algorithmen wurden wiederholt und unabhängig entwickelt, aber Werbos' Algorithmus war der erste mit großer Bedeutung.\footnote{\cite[]{paulwerbosbackpropagation}} Das Prinzip des Algorithmus wird auch heute noch verwendet, es ist dieser Algorithmus der dem Neuronalen Netzwerk das selbstständige Lernen ermöglicht.\footnote{Genaueres in Kapitel \ref{funktionsweise}}

In 1998 veröffentlichte Yann LeCun und sein Team eine Arbeit über die Anwendung eines "`Convolutional Neural Networks\footnote{Ab jetzt als CNN bezeichnet}"' zur Erkennung von geschriebenen Zeichen in einem Dokument.\footnote{\cite[]{cnnhistory}} Diese Arbeit gilt als Ursprung des, für beispielsweise Bilderkennungs Software gut geeignete, CNNs und Weiterentwicklungen werden auch heute noch verwendet.

Obwohl ein großes Potenzial erkannt wurde, war es über die nächsten Jahre wieder recht still. Der nächste große Durchbruch passierte in 2012 als Geoffrey Hinton ein Modell entwickelte, was die Fehlerquote in einer öffentlichen Challenge für Bilderkennung beinahe halbierte.\footnote{\cite[]{geoffrey}} Der Grund dafür waren mehrere fundamentale Neuerungen aus dem Bereich Deep Learning; die wahrscheinlich größte Änderung: Starke Parallelisierung des Backpropagation-Prozesses, durch Verschiebung der Last von der CPU auf die GPU. Aufgrund der starken Überlegenheit eines Grafikprozessors in parallelisierten Prozessen, wie die benötigten Tensormultiplikationen durch die deutlich größere Anzahl an (dafür schwächeren) Kernen im Vergleich zu einer herkömmlichen CPU, kann ein Neuronales Netzwerk mehrere hundertmal schneller trainiert werden.

Heute gibt es (vergleichsweise) simple Frameworks, wie das im Jahr 2015 erschienende TensorFlow oder PyTorch aus 2016, welche das erstellen, trainieren und verwenden von Neuronales Netzwerk enorm vereinfachen. Ihr Funktionsumfang wächst durch die große Open-Source Community ständig.

%Sind die Zitate okay?

\begin{figure}[h]
    \begin{chronology}[10]{1940}{2020}{\textwidth}
        \event{1943}{Erste Arbeit und Experimente}
        \event[1950]{1960}{Bemühungen, ein NN\footnote{Kurzform für "`Neuronales Netzwerk"'} digital umzusetzen}
        \event{1975}{Backpropagation Algorithmus}
        \event{1998}{Erfindung des CNNs}
        \event[2015]{2020}{Entwicklung versch. Frameworks}
    \end{chronology}
    \caption[Zeitstrahl]{Zeitstrahl von 1940 bis 2020 mit den wichtigsten Ereignissen der Entwicklung künstlicher Neuronaler Netzwerke}
\end{figure}

\subsection{Aufbau}

\begin{wrapfigure}{r}{87mm}
    \input{lib/tikz/nn.tex}
    \caption[Aufbau]{Vereinfachter Aufbau eines Neuronales Netzwerk}
\end{wrapfigure}

In Abbildung 2 sieht man den Aufbau eines herkömmlichen künstlichen Neuronalen Netzwerks, so wie es noch vor 40 Jahren verwendet wurde. In der Grafik erkennt man drei Layer mit einer x-beliebigen Anzahl Neuronen, welche untereinander mit jeweils allen Neuronen der vorigen und nächsten Layer verbunden sind. Im Gegensatz zu einem biologischen Neuron, welches nur aktiv oder inaktiv sein kann, kann ein künstliches Neuron einen Zustand in Form eines Wertes von ${0 \leq x \leq 1}$ haben. Jede Verbindung hat einen Weight Paramter und auch jedes Neuron hat einen Bias. Die Anzahl der Hidden Layer kann an das Ziel angepasst und ausgewählt werden und auch die Anzahl der einzelnen Neuronen ist erstmal beliebig, als Faustregel für gute Ergebnisse gilt aber:

\begin{itemize}
    \item Die Anzahl der Neuronen in dem Hidden Layer sollte zwischen der Größe des Input und Output Layers liegen.
    \item Die Anzahl der Neuronen in dem Hidden Layer sollte etwa $\frac{2}{3}$ der Größe des Input Layers plus der Größe des Output Layers entsprechen.
    \item Die Anzahl der Neuronen in einem Hidden Layer sollte weniger als die Hälfte der Größe des Input Layers sein.\footnote{\cite[Alle drei Faustregeln]{heaton}}
\end{itemize}

\subsubsection{Erstellung eines Neuronalen Netzwerks anhand eines Beispiels}

Als Beispiel für ein Neuronales Netzwerk, welches darauf ausgelegt ist, geschriebene Ziffern aus Bildern mit 24x24 Pixeln und nur Graustufen zu erkennen wäre dann: Ein Input Layer mit $24^2$ Neuronen, jeweils für jeden Pixel, welche jeweils eine Aktivierung zwischen 0 (komplett weiß) und 1 (komplett schwarz) haben können, eines. Genau 10 Neuronen im Output Layer, für jedes Zahlzeichen eines. Schließlich muss die Anzahl der Hidden Layer und Neuronen festgelegt werden. Ich wähle als Beispiel 2 Layer mit jeweils 16 Neuronen, die Neuronen-Anzahl kann aber auch unterschiedlich sein. Auch die Weights und Biases werden zunächst zufällig ausgewählt, die Werte werden dann später im Trainingsprozess\footnote{siehe Kapitel \ref{backpropagation}} angepasst.

% \begin{listing}[ht]
%     \begin{minted}[fontsize=\scriptsize,linenos]{python}
%     import tensorflow as tf

%     mnist = tf.keras.datasets.mnist

%     (x_train, y_train), (x_test, y_test) = mnist.load_data()
%     x_train, x_test = x_train / 255.0, x_test / 255.0

%     model = tf.keras.models.Sequential([
%         tf.keras.layers.Flatten(input_shape=(28, 28)),
%         tf.keras.layers.Dense(128, activation='relu'),
%         tf.keras.layers.Dropout(0.2),
%         tf.keras.layers.Dense(10)
%     ])

%     predictions = model(x_train[:1]).numpy()

%     tf.nn.softmax(predictions).numpy()

%     loss_fn = tf.keras.losses.SparseCategoricalCrossentropy(from_logits=True)

%     loss_fn(y_train[:1], predictions).numpy()

%     model.compile(optimizer='adam',
%         loss=loss_fn,
%         metrics=['accuracy'])

%     model.fit(x_train, y_train, epochs=5)

%     model.evaluate(x_test,  y_test, verbose=2)

%     probability_model = tf.keras.Sequential([
%         model,
%         tf.keras.layers.Softmax()
%     ])
%     probability_model(x_test[:5])

%     \end{minted}
%     \caption{Example from numpy}
% \end{listing}

\subsection{Funktionsweise} \label{funktionsweise}

\begin{wrapfigure}{r}{75mm}
    \pgfplotsset{compat=1.16}
\begin{tikzpicture}[declare function={sigma(\x)=1/(1+exp(-\x));
sigmap(\x)=sigma(\x)*(1-sigma(\x));}]
\begin{axis}%
[
    grid=major,     
    xmin=-6,
    xmax=6,
    axis x line=bottom,
    ytick={0,.5,1},
    ymax=1,
    axis y line=middle,
    samples=100,
    domain=-6:6,
    legend style={at={(1,0.9)}}     
]
    \addplot[blue,mark=none]   (x,{sigma(x)});
    \addplot[blue,dotted,mark=none]   (x,{sigmap(x)});
    \legend{$\sigma(x)$,$\sigma'(x)$}
\end{axis}
\end{tikzpicture}
    \caption[Sigmoid]{Die Sigmoidfunktion}
    \label{sigmoid}
\end{wrapfigure}


Ein Neuronales Netzwerk kann man sich eigentlich als eine große Mathematische Funktion vorstellen. In dem zuvor genannten Beispiel wäre es eine Funktion mit 576 Variablen und 10 Ergebnissen. Gibt man dieser Funktion nun ein Bild, beziehungsweise 576 Werte als Input, so werden von links nach rechts alle Weights $w$ und Biases $b$ zusammen mit dem vorigen Aktivierungswerten $a$ berechnet. Da ein Neuron aber nur Werte im Bereich $0\leq x \leq 1$ haben kann, so wird das Ergebniss noch mithilfe einer Aktivierungsfunktion in diesen Bereich umgewandelt. Eine Häufig verwendete Funktion ist dabei die Sigmoidfunktion, siehe Abbildung \ref{sigmoid}.\footnote{\cite{3blue1brown}} Es gibt aber auch noch eine Vielzahl weiterer Funktionen.\footnote{siehe Anhang \ref{anhang:weitereaktivierungsfunktionen}} Die daraus resultierende Funktion würde in etwa so aussehen:\footnote{\cite{3blue1brown}}

\begin{equation}\label{funktion1}
    \sigma(w_1a_1+w_2a_2+w_3a_3+ \ldots +w_na_n+b)
\end{equation}

Um mit dieser Formel alle Aktivierungen auf einmal berechnen zu können verwendet man folgende Funktion, in welcher alle Weights und Biases in Spalten-Vektoren zusammengefasst werden. Die Hochzeichen sind keine Exponenten sondern gelten als Bezeichnung für den Layer, hier beispielsweise 0 und 1. Das Ergebniss dieser Funktion ist ein Vektor mit allen Aktivierungen des darauf folgenden Layers.

\begin{equation}\label{funktion2}
    \sigma
    \begin{pmatrix}
        \begin{bmatrix}
            w_{0,0} & w_{0,1} & \ldots & w_{0,n} \\
            w_{1,0} & w_{1,1} & \ldots & w_{1,n} \\
            \vdots  & \vdots  & \ddots & \vdots  \\
            w_{k,0} & w_{k,1} & \ldots & w_{k,n}
        \end{bmatrix}
        \cdot
        \begin{bmatrix}
            a_0^{(0)} \\a_1^{(0)}\\\vdots\\a_n^{(0)}
        \end{bmatrix}
        +
        \begin{bmatrix}
            b_0 \\b_1\\\vdots\\b_n
        \end{bmatrix}
    \end{pmatrix}
    =
    a^{(1)}
\end{equation}

Auch diese Funktion kann wiederrum kompakter formuliert werden:

\begin{equation}
    \sigma(W\cdot a^{(0)}+b)=a^{(1)}
\end{equation}

Theoretisch wenn ein Neuron einen hohen Aktivierungswert haben soll, wenn beispielsweise eine gerade Linie erkannt wird (um mit anderen Neuronen zusammen im späteren Verlauf aus den Mustern ganze Ziffern zu erkennen), so müssen die Weights der zu dem Neuron führenden Verbindungen alle möglichst niedrige Aktivierungen haben, ausser an den Stellen an denen die Linie sich befinden soll. Um sicherzustellen, dass es sich wirklich um eine gerade Linie handelt befindet sich direkt über dem Strich ein Bereich in dem keine Aktivierungen sein sollten, dieser ist rot markiert. Das erkennt man in Abbildung \ref{examples} sehr gut. In a erkennt man die zu erkennende Linie und in b sieht man die zugehörigen Weights der Input Nodes zu dem Neuron. Dabei stellt grün positive Weights da, rot negative und Weiß/Transparent ist 0. Der Bias des Neurons stellt eine Zusätzliche Hürde oder eine Verstärkung da, was auch in Formel \ref{funktion1} als $b$ sichtbar ist.

\begin{figure}[h]
    \centering
    \subfloat[\centering Zu erkennendes Bild]{\begin{tikzpicture}
    \draw[step=3mm,gray,very thin] (0,0) grid (4.2,4.2);
    \filldraw[fill=black, draw=black] (1.2,1.2) rectangle (3,1.5);
    \filldraw[fill=none, draw=red!50!white] (1.5,1.5) rectangle (3,1.8);
    \filldraw[fill=none, draw=red, dotted] (1.2,1.5) rectangle (1.5,1.8);
\end{tikzpicture}}%
    \qquad
    \subfloat[\centering Benötigte Weights]{\begin{tikzpicture}
    \draw[step=3mm,gray,very thin] (0,0) grid (4.2,4.2);
    \filldraw[fill=green!50!white, draw=none] (1.2,1.2) rectangle (3,1.5);
    \filldraw[fill=red!50!white, draw=none] (1.5,1.5) rectangle (3,1.8);
    \filldraw[fill=red!25!white, draw=none] (1.2,1.5) rectangle (1.5,1.8);
\end{tikzpicture}}
    \caption[Visualisierung]{Visualisierung der gewünschten Formen a und die dazugehörigen Weights b (jeweils abgeschnitten)}
    \label{examples}%
\end{figure}

%\begin{tikzpicture}
    \begin{axis}[domain=-1:1, y domain=-1:1,  colormap/hot,samples=32]
        \addplot3[surf, fill opacity=0.7,] {x*y*exp(x+2*y-9*x^2-9*y^2)};
    \end{axis}
\end{tikzpicture}

\subsubsection{Trainieren - Backpropagation} \label{backpropagation}

\section{Labelcheck als Smartphone App}\label{labelcheck}

In diesem Kapitel wird mithilfe von Python und TensorFlow ein Netzwerk erstellt und trainiert, sowie anschließend eine mobile App mit Dart und Flutter entwickelt, welche dann öffentlich für den Download bereit stehen soll.

\subsection{Die Idee}

Die Ziel ist es, dass die App es ermöglicht im Supermarkt die verschiedenen Label der Produkte zu scannen und dem Nutzer dann Auskunft über die Vertrauenswürdigkeit und generelle Aussage des Labels gibt. Die Idee habe ich in meinem Erdkunde Leistungskurs bekommen als wir über die Problematik gestoßen sind, dass die Bedeutung der verschiedenen Label eher untransparent gegenüber dem Verbraucher ist. Ich habe alle Siegel einmal im Angang \ref{angang:label} vorgestellt, da sie nur indirekt mit dem Thema der Facharbeit zusammenhängen.

\subsection{Erstellen des Models}\label{erstellen des modells}

Zum erstellen und trainieren des Modells werde ich die Sprache Python und das Framework TensorFlow verwenden. 

\subsubsection{Trainieren des Modells mit TensorFlow und Python}

\emph{Vollständiger Code in meinem Colab Notebook: \url{https://bit.ly/34Ggfuh}\footnote{Ungekürzter Link: \url{https://colab.research.google.com/drive/1ty_QQlL038YT6KpBjSdqGvIGyH0YXwxW}} und im Anhang \ref{anhang:labelchecktf}}

\subsubsection{Ergebnisse des Models}

\subsection{Entwickeln der App}

Zum entwickeln der App verwende ich die Sprache Dart und das zugehörige Framework Flutter. Im Gegensatz zu nativ geschriebenen Apps bietet Flutter die möglichkeit nur einmal den Code in Dart zu schreiben und anschließend kann die App für alle großen Platformen kompiliert werden, dazu zählen iOS, Android, aber auch Linux, Windows, MacOS und Web/Javascript. Nun muss die App ja Zugriff auf die Kamera haben und auch in die Supermärkte "`mitgebracht"' werden, weshalb nur iOS und Android relevant sind. Die App ist vollständig Quelloffen und kann auf GitHub unter \url{https://github.com/phibr0/labelcheck} eingesehen werden.

\subsubsection{Das Framework: Flutter}

Flutter Apps funktionieren anders als nativ entwickelte Apps. Herkömmliche native Apps verwenden die UI\footnote{User Interface; deutsch: Benutzeroberfläche} Komponenten des Betriebssystems und sehen daher auf jedem Gerät mit unterschiedlichen Betriebssystemversionen leicht unterschiedlich aus. Flutter hingegen stellt ein "`Canvas"' Element bereit, welches als unterliegende Grafik-Engine Google's Skia nutzt.\footnote{\cite{flutterarchitecture}; mehr dazu im Anhang \ref{anhang:flutterarc}} In Flutter stehen eine Menge UI Komponenten zur verfügung die entweder Googles Material Design guidelines oder Apples Human interface guidelines folgen. Der Dart Code stellt dann als Einzigen Eintrittspunkt die \mintinline{Dart}{main()} Methode bereit, aus welchem dann die App gestartet wird. Das Framework wird mit der Methode \mintinline{Dart}{runApp(Widget)} initialisert. In Flutter ist jedes UI Element ein "`Widget"', wodurch sich dann in Kombination in einer App große Widgethierarchien erstellen lassen. Der Code einer simplen App, welche nur den Text "`Hello World!"' in der Mitte des Bildschirms anzeigen würde, sehe demnach so aus:

\begin{wrapfigure}{l}{75mm} 
    \begin{minted}[fontsize=\footnotesize,linenos]{Dart}
import 'package:flutter/widgets.dart';

void main() => runApp(MyApp());

class MyApp extends StatelessWidget {
  @override
  Widget build(BuildContext context) {
    return Center(
      child: Text('Hello World!'),
    );
  }
}
\end{minted}
\end{wrapfigure}

Zeile 1: Importieren der Widgets aus dem Framework zum bereitstellen der Klassen, wie \mintinline{Dart}{Stateless}- und \mintinline{Dart}{StatefulWidget} und den Methoden, wie \mintinline{Dart}{runApp()}.

Zeile 3: Die \mintinline{Dart}{main()} Methode mit dem einzigen Aufruf \mintinline{Dart}{runApp()}, was die Klasse \mintinline{Dart}{MyApp} als Flutter App initialisert.

Zeile 5f.: \mintinline{Dart}{MyApp} erbt die Klasse \mintinline{Dart}{StatelessWidget}, überschreibt die \mintinline{Dart}{build()} Methode und gibt eine Widgethierarchie zurück.

Zeile 8f.: Das \mintinline{Dart}{Center} Widget nimmt als einzigen benannten Parameter ein weiteres (child) Widget an, welches in diesem Fall ein \mintinline{Dart}{Text} Widget ist. Stateless bedeutet hier, dass sich der Zustand des Widgets nicht während der Laufzeit verändern kann. Im Gegensatz dazu gibt es auch noch \mintinline{Dart}{StatefulWidget}'s welche die Möglichkeit haben bei Bedarf das UI zu "`rebuilden"'.

Um der App das Aussehen von nativen Apps zu verleihen, verwendet man üblicherweise ein \mintinline{Dart}{MaterialApp} (Material Design / Android) oder \mintinline{Dart}{CupertinoApp} (Human Interface / iOS) Widget, was zudem noch wichtige Variablen, wie \mintinline{Dart}{ThemeData} für verschiedene Farben, die in der App einfach verwendet werden können oder \mintinline{Dart}{LocalizationsDelegate} welche für die Bereitstellung verschiedener Sprachen gebraucht werden, beinhaltet.

\subsubsection{Importieren des Models}

Es gibt eine speziell für mobile Geräte angepasste Version von TensorFlow mit dem Namen TensorFlow Lite. TFLite hat einen geringeren Speicherbedarf kann dafür aber auch weniger als das herkömmliche Framework.\footnote{\cite{tflite}} Implementationen dafür gibt es allerdings nicht in Dart, stattdessen verwendet man PlatformChannel's in Flutter, welche die Möglichkeit bieten platformspezifischen Code aus einer Flutter App auszuführen (Java/Kotlin für Android und Objective-C/Swift für iOS).\footnote{\cite{flutterplatformcode}}

Desweiteren muss das zuvor erstellte TensorFlow Model in ein TensorFlow Lite Model umgewandelt werden.

\subsubsection{Funktionsweise der App}



\subsubsection{Veröffentlichen der App}

Ich habe die App im Google PlayStore veröffentlicht\footnote{\url{https://play.google.com/store/apps/details?id=com.phillip.labelcheck}}, theoretisch wäre es auch möglich die App für iOS im Appstore anzubieten, leider ist ein Apple Entwicklerkonto aber mit höheren und jährlichen Kosten verbunden.

Auch musste ich eine Datenschutzerklärung bereitstellen, sie ist unter \url{https://labelcheck.phibr0.de} erreichbar. 

\section{Fazit}

Ich habe in den letzen Monaten viel über die App entwicklung mit Flutter gelernt, leider habe ich mit dieser App schon etwas früher begonnen, weswegen ich heute wahrscheinlich viele Dinge anders gemacht hätte. Ich habe nicht viel darauf geachtet den UI Code von dem Logik Code zu trennen und jetzt kommt es häufig vor, dass ich lange nach etwas suchen muss. Ich plane allerdings dies noch zu beheben. Auch habe ich viele Variablen als dynamisch deklariert, was die Autovervollständigung beeinträchtigt und daher behoben werden sollte. Das beeinträchtigt allerdings nicht die Funktionalität, beziehungsweise den Nutzer sondern nur mich, solange ich noch weiter an der App arbeiten möchte. Desweiteren ist Dart in der neusten Version nun standardmäßig "`Null-Safe"'\footnote{Damit Variablen einen \mintinline{Dart}{null} Wert haben können, müssen sie speziell deklariert werden. Dies fängt viele Fehler bei Laufzeit ab.}, weswegen ich meine App auf diese Version manuell migrieren muss.

Dennoch konnte ich folgende Funktionen erfolgreich implementieren:

\begin{itemize}
  \item Vollständige Übersetzungen für Englisch und Deutsch, sowie automatischem Anpassen an die Systemsprache
  \item Modernes Design nach Material Design Guidelines
  \begin{itemize}
    \item Automatischer Wechsel zwischen hellem und dunklem Modus
  \end{itemize}
  \item Automatisches Sammeln von anonymisierten Nutzerstatistiken und Absturzberichten durch Google Analytics/Firebase Crashlytics
  \begin{itemize}
    \item Zusätzlich manuelle Fehlerberichterstattung durch den Nutzer via E-Mail
  \end{itemize}
  \item Integration der Wikipedia API für noch mehr Informationen über das Label
  \item Klassifizieren von Fotos, entweder in der App aufgenommen oder aus einer Datei, mit (!TODO!)\% Genauigkeit
  \item Hosten einer Website für die Datenschutzerklärung und Nutzungsbedingungen
\end{itemize}

Auch habe ich einen größeren Einblick in die unterliegende Mathematik von Neuronalen Netzen erhalten. Alles in einem würde ich sagen die Facharbeit war für mich ein Erfolg.

\newpage
\appendix
\label{Anhang}
\section{Anhang}

\subsection{Weitere Aktivierungsfunktionen}\label{anhang:weitereaktivierungsfunktionen}

\begin{figure}[h]
    \center
    \begin{tikzpicture}
    \begin{axis}
        [
            grid=major,
            xmin=-3,
            xmax=3,
            axis x line=bottom,
            ytick={0,1,2},
            ymax=2.5,
            ymin=-0.5,
            axis y line=middle,
            legend style={at={(0.5,-0.2)},anchor=north}
        ]
        \addplot[red,mark=none]{(x>=0)*x};
        \addplot[blue, mark=none]{0.5*(tanh(\x)+1)};
        \legend{ReLU - $\max{(0,x)}$, (Verschobene) Hyperbolische Tangente - $0.5(\tanh{(x)}+1)$}
    \end{axis}
\end{tikzpicture}
    \caption[Aktivierungsfunktionen]{Weitere Aktivierungsfunktionen, ergänzend zu der Sigmoid Funktion aus Kapitel \ref{funktionsweise}}
    \label{Aktivierungsfunktionen}%
\end{figure}



\newpage
\printbibliography[heading=bibintoc, title={Literaturverzeichnis}]

\end{document}